\begin{center}
{\large {\bf  ABSTRACT }}

\end{center}  


 
%  Extracting signals which is in a small cross-section compared to the large cross-section of the background processes has been helped by introducing machine learning (ML) techniques for classification purposes. Classification algorithms in machine learning is a type of supervised learning where the outputs are constrained only to a limited set of values or classes such as signals or backgrounds. In this thesis, the classification of the produced heavy resonances from  collision at the Large Hadron Collider(LHC) is presented. The machine learning technique, deep neural network(DNN), is used to classify the resonances from the background processes. Heavy resonances such as Vector-like quark(VLQ), Tprime () at different mass points [600, 1200] GeV are used as the signal while Standard model Higgs(SMH) and Non-Resonant background(NRB) as the backgrounds for the training and testing of the DNN model. On the DNN output variable, the expected limit has been calculated using Higgs Combined Tools under the  environment. Preliminary studies shows DNN can have a better/comparable results to other machine learning techniques used for the high energy physics analysis. 
%  DNN seems to sensitivity in signal-background separations and the expected limit than the other machine learning techniques Boosted Decision Trees(BDT). The DNN and BDT performance have been used to search H $\longrightarrow$ $\gamma\gamma$ in the leptonic channel.

Extracting signals in a small cross-section compared to the large cross-section of the background processes has been helped by introducing machine learning (ML) techniques for classification purposes. Classification algorithms in machine learning is a type of supervised learning where the outputs are constrained only to a limited set of values or classes such as signals or backgrounds. This thesis presents the classification of the produced heavy resonances from $pp$ collision at the Large Hadron Collider(LHC). The machine learning technique, deep neural network(DNN), is used to classify the resonances from the background processes. Heavy resonances such as Vector-like quark(VLQ), Tprime ($T'$) at different mass points [600, 1200] GeV are used as the signal while Standard model Higgs(SMH) and Non-Resonant background(NRB) as the backgrounds for the training and testing of the DNN model. On the DNN output variable, the expected limit has been calculated using Higgs Combined Tools under the \texttt{CMSSW} environment. Preliminary studies show DNN can have a better/comparable results to other machine learning techniques used for the high-energy physics analysis.