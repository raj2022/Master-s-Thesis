Naturalness arguments [1] suggest there
should be a mechanism that cancels out the quadratically
divergent contributions to the Higgs boson mass caused by
radiative corrections from standard model (SM) particles.
Several explanations are proposed in theories beyond the
SM. Little Higgs [2,3] and composite Higgs [4,5] models
introduce a spontaneously broken global symmetry, with
the Higgs boson emerging as a pseudo Nambu-Goldstone
boson [6]. Such models predict the existence of vectorlike
quarks (VLQs), color-triplet spin-1=2 fermions whose leftand right-handed chiralities transform in the same way
under weak isospin [7,8]. In these models, VLQs are
expected to couple preferentially to third-generation quarks
[7,9] and can have flavor-changing neutral-current decays
in addition to charged-current decays. An up-type VLQ T
with charge þ2=3 can decay into Wb, Zt, or Ht. Similarly,
a down-type quark B with charge −1=3 can decay into Wt,
Zb, or Hb. In order to be consistent with results from
precision electroweak measurements, the mass-splitting
between VLQs belonging to the same SU(2) multiplet is
required to be small [10], forbidding cascade decays such
as T → WB. Couplings between the VLQs and the firstand second-generation quarks, although not favored, are
not excluded [11,12].
At the Large Hadron Collider (LHC), VLQs with masses
below approximately 1 TeV would mainly be pair produced, a process dominated by the strong interaction. The
corresponding predicted cross section ranges from 195
to 2.0 fb for quark masses from 800 to 1500 GeV [13]
and depends only on the quark mass. Production of single
VLQs via the electroweak interaction is also possible,
but depends on the strength of the interaction between
the new quarks and the weak gauge bosons. Representative
Feynman diagrams for BB¯ and TT¯ production and decay are shown in Fig. 1

The branching ratio (B) for each decay mode
(T → Wb; Zt; Ht and B → Wt; Zb; Hb) depends on the
VLQ mass and weak-isospin quantum numbers, as calculated in Ref. [8]. For a singlet T, all three decay modes have
sizable branching ratios, while the charged-current decay
mode T → Wb is absent if T is either in a ðX; TÞ doublet,
where X is a VLQ with a charge of þ5=3, or in a ðT;BÞ
doublet with jVTbj ≪ jVtBj, where Vij are the elements
of a generalized Cabibbo-Kobayashi-Maskawa matrix
[8,14,15]. Since the T quark branching ratios are identical
in both doublets, no distinction is made between them when
referring to the doublet T results. A singlet B will have a
sizable branching ratio to all three decay channels, while
the branching ratios in the doublet case depend on whether
it is in a ðT;BÞ doublet or ðB; YÞ doublet, where Y is a VLQ
with a charge of −4=3. For a ðB; YÞ doublet, only neutral
current couplings to SM quarks are allowed at leading order
(LO), so the B → Wt decay is forbidden. Conversely, for a
ðT;BÞ doublet with jVTbj ≪ jVtBj, B → Wt is the only
allowed decay. Therefore, the specific B doublet scenario
will be stated when interpreting the result.


—Searches for pair-produced
VLQ partners of the third-generation quarks have been
performed by ATLAS [16–22] and CMS [23–25] at the









\begin{itemize}
    \item J. A. Aguilar-Saavedra, Identifying top partners at LHC,
J. High Energy Phys. 11 (2009) 030.
\item  L. Susskind, Dynamics of spontaneous symmetry breaking
in the Weinberg-Salam theory, Phys. Rev. D 20, 2619
(1979).
\item N. Arkani-Hamed, A. G. Cohen, E. Katz, and A. E. Nelson,
The littlest Higgs, J. High Energy Phys. 07 (2002) 034.
\item  D. B. Kaplan, H. Georgi, and S. Dimopoulos, Composite
Higgs scalars, Phys. Lett. B 136B, 187 (1984).
\item K. Agashe, R. Contino, and A. Pomarol, The minimal
composite Higgs model, Nucl. Phys. B719, 165 (2005).


[16] ATLAS Collaboration, Search for pair production of uptype vector-like quarks and for four-topquark events in final
states with multiple b-jets with the ATLAS detector, J. High
Energy Phys. 07 (2018) 089.
[17] ATLAS Collaboration, Search for pair production of heavy
vector-like quarks decaying to high-pT W bosons and b
quarks in the lepton-plus-jets final state in pp collisions at ffiffi
s p ¼ 13 TeV with the ATLAS detector, J. High Energy
Phys. 10 (2017) 141.
[18] ATLAS Collaboration, Search for pair production of heavy
vector-like quarks decaying into high-pT W bosons and top
quarks in the lepton-plus-jets final state in pp collisions at ffiffi
s p ¼ 13 TeV with the ATLAS detector, J. High Energy
Phys. 08 (2018) 048.
[19] ATLAS Collaboration, Search for pair production of
vector-like top quarks in events with one lepton, jets, and
missing transverse momentum in ffiffi
s p ¼ 13 TeV pp collisions with the ATLAS detector, J. High Energy Phys. 08
(2017) 052.
[20] ATLAS Collaboration, Search for pair- and singleproduction of vector-like quarks in final states with at least
one Z boson decaying into a pair of electrons or muons in
pp collision data collected with the ATLAS detector at ffiffi
s p ¼ 13 TeV, arXiv:1806.10555.
[21] ATLAS Collaboration, Search for new phenomena in events
with same-charge leptons and b-jets in pp collisions at ffiffi
s p ¼ 13 TeV with the ATLAS detector, arXiv:1807.11883.
[22] ATLAS Collaboration, Search for pair production of heavy
vector-like quarks decaying into hadronic final states in pp
collisions at ffiffi
s p ¼ 13 TeV with the ATLAS detector, arXiv:
1808.01771.
[23] CMS Collaboration, Search for pair production of vectorlike T and B quarks in single-lepton final states using
boosted jet substructure in proton-proton collisions at ffiffi
s p ¼ 13 TeV, J. High Energy Phys. 11 (2017) 085.
[24] CMS Collaboration, Search for pair production of vectorlike quarks in the bWbW¯ channel from proton-proton
collisions at ffiffi
s p ¼ 13 TeV, Phys. Lett. B 779, 82 (2018).
[25] CMS Collaboration, Search for vector-like T and B quark
pairs in final states with leptons at ffiffi
s p ¼ 13 TeV, J. High
Energy Phys. 08 (2018) 177.




\end{itemize}