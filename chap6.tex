\chapter{\label{summary}Summary and Conclusion}
In this project, machine learning techniques(Deep Neural Network) have been explored to classify data from CMS, CERN as the signal events from the backgrounds. We mainly focused on simulated Monte Carlo data samples produced after PP collision at the large hadron collider(LHC) with center of mass energy ($\sqrt{s}$) 13TeV at the integrated luminosity of 137 $fb^{-1}$. A few of the samples produced are Tprime $T'$, Standard model Higgs(SMH), and the Non-Resonant backgrounds(NRB) used for the analysis. The DNN model training has been modeled for three different ranges of mass points of $T'$([600,700]$\cup$[800, 1000]$\cup$ [1100, 1200]) as the signals while the SMH as the the backgrounds for the better sensitivity of the DNN model. The advantages of using deep neural networks (DNN) for training  as it updates all the parameters itself to give a better optimized outputs by updating its weights using back propagation methods after each iteration or the epochs.

In this analysis, extensively studied multivariate techniques to separate arbitrary signal (T$'$) from the standard model Higgs(SMH) backgrounds processes using the CMS detector simulation and the reconstruction program. In the machine learning training, model separation of signals and backgrounds seems to be getting better with increasing mass of the $T'$.  The DNN training output score used to tests over the different simulated datasets and plotted as the stacked ratio plot. To scale the data and the Monte Carlo, the Monte Carlo samples have been scaled with corresponding outputs from the linear fit of the ratio plot.  

In this blinded analysis with 115 < $m_{\gamma\gamma}< 135$GeV, with the cuts applied to separate signal events from the backgrounds and further scaling the NRB backgrounds with scaling parameters, all the Monte Carlo were rescaled. The performance of DNN has been well evaluated over the Non Resonant background processes. Using Higgs Combined Tools under CMMSW environment, the expected limit at 95\% CL on $T'$ production processes have been extracted at each $T'$ mass in the range
[600,1200] using the DNN based selection criteria, as well as for the present ongoing matured analysis at the CMS analysis. 

In conclusion, we can say that the DNN-based analysis sensitivity is yet to reach the exclusion potential. The DNN based sensitivity 
% searched for single produced vector-like T quarks decaying at $\sqrt{s}$ = 13 TeV. Here, an analysis to look for $T{^'} \longrightarrow$ top + H with H$\longrightarrow \gamma \gamma$. This analysis is composed on leptonic channel only and is based on DNNs to seperate the signal $T^'$ from the major backgrounds. Expected limits have been calculated for the DNN output variables. The predicted limit output from the DNN, compared with the expected limits of the another machine learning techniques, BDT, 

% By the end of 2025, LHC is supposed to be upgraded to high luminosity LHC(HL-LHC) which is supposed to have collected data with an integrated luminosity of 3000 $fb^{-1}$. With the higher luminosity, a new analysis can be performed using sophisticated machine learning techniques to have more signal significance. With more data, we hope to discover some exciting new physics!
\setcounter{equation}{0}
\setcounter{table}{0}
\setcounter{figure}{0}
%\baselineskip 24pt


    




